%!TEX TS-program = xelatex
%!TEX encoding = UTF-8 Unicode
% Awesome CV LaTeX Template for CV/Resume
%
% This template has been downloaded from:
% https://github.com/posquit0/Awesome-CV
%
% Author:
% Claud D. Park <posquit0.bj@gmail.com>
% http://www.posquit0.com
%
% Template license:
% CC BY-SA 4.0 (https://creativecommons.org/licenses/by-sa/4.0/)
%


%-------------------------------------------------------------------------------
% CONFIGURATIONS
%-------------------------------------------------------------------------------
% A4 paper size by default, use 'letterpaper' for US letter
\documentclass[11pt, a4paper, UTF8]{awesome-cv}

% Chinese
\usepackage[UTF8, heading = false, scheme = plain]{ctex}

% Configure page margins with geometry
\geometry{left=1.4cm, top=.8cm, right=1.4cm, bottom=1.8cm, footskip=.5cm}

% Specify the location of the included fonts
\fontdir[fonts/]

% Color for highlights
% Awesome Colors: awesome-emerald, awesome-skyblue, awesome-red, awesome-pink, awesome-orange
%                 awesome-nephritis, awesome-concrete, awesome-darknight
\colorlet{awesome}{awesome-skyblue}
% Uncomment if you would like to specify your own color
% \definecolor{awesome}{HTML}{CA63A8}

% Colors for text
% Uncomment if you would like to specify your own color
% \definecolor{darktext}{HTML}{414141}
% \definecolor{text}{HTML}{333333}
% \definecolor{graytext}{HTML}{5D5D5D}
% \definecolor{lighttext}{HTML}{999999}

% Set false if you don't want to highlight section with awesome color
\setbool{acvSectionColorHighlight}{false}

% If you would like to change the social information separator from a pipe (|) to something else
\renewcommand{\acvHeaderSocialSep}{\quad\textbar\quad}

%%%%%%%%%%%%%%%%%% custom settings

\geometry{left=2.0cm, top=1.5cm, right=2.0cm, bottom=2.0cm, footskip=.5cm}

\renewcommand{\acvSectionTopSkip}{0mm}
\renewcommand{\acvSectionContentTopSkip}{1.5mm}

\renewcommand{\acvHeaderAfterNameSkip}{1mm}
\renewcommand{\acvHeaderAfterSocialSkip}{1mm}

\renewcommand*{\paragraphstyle}{\fontsize{10pt}{1em}\bodyfontlight\upshape\color{text}}

%\renewcommand*{\headerfirstnamestyle}[1]{{\fontsize{32pt}{1em}\headerfontlight\color{graytext} #1}}
\renewcommand*{\headerlastnamestyle}[1]{{\fontsize{26pt}{1em}\headerfont\bfseries\color{text} #1}}
\renewcommand*{\headerpositionstyle}[1]{{\fontsize{10pt}{1em}\bodyfont\itshape\color{awesome} #1}}
\renewcommand*{\headeraddressstyle}[1]{{\fontsize{9pt}{1em}\headerfont\itshape\color{text} #1}}
\renewcommand*{\headersocialstyle}[1]{{\fontsize{9pt}{1em}\headerfont\color{text} #1}}
\renewcommand*{\headerquotestyle}[1]{{\fontsize{9pt}{1em}\bodyfont\itshape\color{text} #1}}

\renewcommand*{\headerlinestyle}[1]{{\fontsize{10pt}{1em}\bodyfontlight\color{text} #1}}%\bodyfont\itshape

\renewcommand*{\entrytitlestyle}[1]{{\fontsize{11pt}{1em}\bodyfont\bfseries\color{darktext} #1}}
\renewcommand*{\entrypositionstyle}[1]{{\fontsize{9pt}{1em}\bodyfont\bfseries\color{graytext} #1}}%\itshape
\renewcommand*{\entrydatestyle}[1]{{\fontsize{9pt}{1em}\bodyfont\itshape\color{graytext} #1}}
\renewcommand*{\entrylocationstyle}[1]{{\fontsize{9pt}{1em}\bodyfontlight\itshape\color{text} #1}}
\renewcommand*{\descriptionstyle}[1]{{\fontsize{10pt}{1em}\bodyfontlight\upshape\color{text} #1}}

\renewcommand*{\skilltypestyle}[1]{{\fontsize{10pt}{1em}\bodyfont\bfseries\color{darktext} #1}}
\renewcommand*{\skillsetstyle}[1]{{\fontsize{10pt}{1em}\bodyfontlight\color{text} #1}}

%%%%%%%%%%%%%%%%%% custom settings

%-------------------------------------------------------------------------------
%	PERSONAL INFORMATION
%	Comment any of the lines below if they are not required
%-------------------------------------------------------------------------------
% Available options: circle|rectangle,edge/noedge,left/right
%\photo[circle,noedge,left]{./profile}
\name{}{Jian Yang}
%\position{Software Architect{\enskip\cdotp\enskip}Security Expert}
\position{Senior Data Engineer{\enskip\cdotp\enskip}Shanghai}
%\address{42-8, Bangbae-ro 15-gil, Seocho-gu, Seoul, 00681, Rep. of KOREA}

\mobile{(+86) 18516191607}
\email{ernestyj@outlook.com}
%\homepage{www.posquit0.com}
\github{Ernestyj}
%\linkedin{posquit0}
% \gitlab{gitlab-id}
% \stackoverflow{SO-id}{SO-name}
% \twitter{@twit}
% \skype{skype-id}
% \reddit{reddit-id}
 %\medium{madium-id}
% \googlescholar{googlescholar-id}{name-to-display}
%% \firstname and \lastname will be used
% \googlescholar{googlescholar-id}{}
%\extrainfo{电话: (+86) 18516191607{\quad\textbar\quad}邮件: ernestyj@outlook.com}

%\quote{Master{\ \ \cdotp\ \ }Shanghai Jiao Tong University{\quad\textbar\quad}Bachelor{\ \ \cdotp\ \ }University of Electronic Science and Technology of China}
%\linea{Major{\ \ \cdotp\ \ }Software Engineering}
%\lineb{M.Eng. in Software Engineering, Shanghai Jiao Tong University}
%\linec{B.Eng. in Software Engineering, University of Electronic Science and Technology of China}


%-------------------------------------------------------------------------------
\begin{document}

% Print the header with above personal informations
% Give optional argument to change alignment(C: center, L: left, R: right)
\makecvheader[R]

% Print the footer with 3 arguments(<left>, <center>, <right>)
% Leave any of these blank if they are not needed
\makecvfooter
  {\ }
  {\ }
  %{\thepage}
  {\ }


%-------------------------------------------------------------------------------
%	CV/RESUME CONTENT
%	Each section is imported separately, open each file in turn to modify content
%-------------------------------------------------------------------------------

%-------------------------------------------------------------------------------
%	SECTION TITLE
%-------------------------------------------------------------------------------
\cvsection{Summary}

%-------------------------------------------------------------------------------
%	CONTENT
%-------------------------------------------------------------------------------
\begin{cvparagraph}
%---------------------------------------------------------
%三年工作经验。参与开发基于C++的大型分布式NoSQL数据库,熟悉C++面向对象开发。参与设计、开发金融交易风控系统,包括决策规则分析系统、自动化风控平台、实时交易数据监控报警平台;熟悉Python全栈开发,数据分析、建模与可视化。
Current senior data engineer at Paypal with around 3 years work experience. Developed C++ based large distributed NoSQL database; experienced with C++ developing. Designed and developed financial transaction risk control system, including decision rules analysis, automatic risk control platform, real time transaction data monitoring and alerting; experienced with Python full stack development, data analysis, model building and visualization.
\end{cvparagraph}



%%-------------------------------------------------------------------------------
%	SECTION TITLE
%-------------------------------------------------------------------------------
\cvsection{Work Experience}


%-------------------------------------------------------------------------------
%	CONTENT
%-------------------------------------------------------------------------------
\begin{cventries}

%---------------------------------------------------------
  \cventry
    {Software Architect} % Job title
    {Omnious. Co., Ltd.} % Organization
    {Seoul, S.Korea} % Location
    {Jun. 2017 - May. 2018} % Date(s)
    {
      \begin{cvitems} % Description(s) of tasks/responsibilities
        \item {Provisioned an easily managable hybrid infrastructure(Amazon AWS + On-premise) utilizing IaC(Infrastructure as Code) tools like Ansible, Packer and Terraform.}
        \item {Built fully automated CI/CD pipelines on CircleCI for containerized applications using Docker, AWS ECR and Rancher.}
        \item {Designed an overall service architecture and pipelines of the Machine Learning based Fashion Tagging API SaaS product with the micro-services architecture.}
        \item {Implemented several API microservices in Node.js Koa and in the serverless AWS Lambda functions.}
        \item {Deployed a centralized logging environment(ELK, Filebeat, CloudWatch, S3) which gather log data from docker containers and AWS resources.}
        \item {Deployed a centralized monitoring environment(Grafana, InfluxDB, CollectD) which gather system metrics as well as docker run-time metrics.}
      \end{cvitems}
    }

%---------------------------------------------------------
  \cventry
    {Co-founder \& Software Engineer} % Job title
    {PLAT Corp.} % Organization
    {Seoul, S.Korea} % Location
    {Jan. 2016 - Jun. 2017} % Date(s)
    {
      \begin{cvitems} % Description(s) of tasks/responsibilities
        \item {Implemented RESTful API server for car rental booking application(CARPLAT in Google Play).}
        \item {Built and deployed overall service infrastructure utilizing Docker container, CircleCI, and several AWS stack(Including EC2, ECS, Route 53, S3, CloudFront, RDS, ElastiCache, IAM), focusing on high-availability, fault tolerance, and auto-scaling.}
        \item {Developed an easy-to-use Payment module which connects to major PG(Payment Gateway) companies in Korea.}
      \end{cvitems}
    }

%---------------------------------------------------------
  \cventry
    {Software Engineer \& Security Researcher (Compulsory Military Service)} % Job title
    {R.O.K Cyber Command, MND} % Organization
    {Seoul, S.Korea} % Location
    {Aug. 2014 - Apr. 2016} % Date(s)
    {
      \begin{cvitems} % Description(s) of tasks/responsibilities
        \item {Lead engineer on agent-less backtracking system that can discover client device's fingerprint(including public and private IP) independently of the Proxy, VPN and NAT.}
        \item {Implemented a distributed web stress test tool with high anonymity.}
        \item {Implemented a military cooperation system which is web based real time messenger in Scala on Lift.}
      \end{cvitems}
    }

%---------------------------------------------------------
  \cventry
    {Game Developer Intern at Global Internship Program} % Job title
    {NEXON} % Organization
    {Seoul, S.Korea \& LA, U.S.A} % Location
    {Jan. 2013 - Feb. 2013} % Date(s)
    {
      \begin{cvitems} % Description(s) of tasks/responsibilities
        \item {Developed in Cocos2d-x an action puzzle game(Dragon Buster) targeting U.S. market.}
        \item {Implemented API server which is communicating with game client and In-App Store, along with two other team members who wrote the game logic and designed game graphics.}
        \item {Won the 2nd prize in final evaluation.}
      \end{cvitems}
    }

%---------------------------------------------------------
  \cventry
    {Software Engineer} % Job title
    {ShitOne Corp.} % Organization
    {Seoul, S.Korea} % Location
    {Dec. 2011 - Feb. 2012} % Date(s)
    {
      \begin{cvitems} % Description(s) of tasks/responsibilities
        \item {Developed a proxy drive smartphone application which connects proxy driver and customer.}
        \item {Implemented overall Android application logic and wrote API server for community service, along with lead engineer who designed bidding protocol on raw socket and implemented API server for bidding.}
      \end{cvitems}
    }

%---------------------------------------------------------
  \cventry
    {Freelance Penetration Tester} % Job title
    {SAMSUNG Electronics} % Organization
    {S.Korea} % Location
    {Sep. 2013, Mar. 2011 - Oct. 2011} % Date(s)
    {
      \begin{cvitems} % Description(s) of tasks/responsibilities
        \item {Conducted penetration testing on SAMSUNG KNOX, which is solution for enterprise mobile security.}
        \item {Conducted penetration testing on SAMSUNG Smart TV.}
      \end{cvitems}
      %\begin{cvsubentries}
      %  \cvsubentry{}{KNOX(Solution for Enterprise Mobile Security) Penetration Testing}{Sep. 2013}{}
      %  \cvsubentry{}{Smart TV Penetration Testing}{Mar. 2011 - Oct. 2011}{}
      %\end{cvsubentries}
    }

%---------------------------------------------------------
\end{cventries}

%-------------------------------------------------------------------------------
%	SECTION TITLE
%-------------------------------------------------------------------------------
\cvsection{Work Experience}
%-------------------------------------------------------------------------------
%	CONTENT
%-------------------------------------------------------------------------------
\begin{cventries}
%---------------------------------------------------------
%  \cventry
%    {Software Architect} % Job title
%    {Omnious. Co., Ltd.} % Organization
%    {Seoul, S.Korea} % Location
%    {Jun. 2017 - May. 2018} % Date(s)
%    {
%      \begin{cvitems} % Description(s) of tasks/responsibilities
%        \item {Provisioned an easily managable hybrid infrastructure(Amazon AWS + On-premise) utilizing IaC(Infrastructure as Code) tools like Ansible, Packer and Terraform.}
%        \item {Built fully automated CI/CD pipelines on CircleCI for containerized applications using Docker, AWS ECR and Rancher.}
%        \item {Designed an overall service architecture and pipelines of the Machine Learning based Fashion Tagging API SaaS product with the micro-services architecture.}
%        \item {Implemented several API microservices in Node.js Koa and in the serverless AWS Lambda functions.}
%        \item {Deployed a centralized logging environment(ELK, Filebeat, CloudWatch, S3) which gather log data from docker containers and AWS resources.}
%        \item {Deployed a centralized monitoring environment(Grafana, InfluxDB, CollectD) which gather system metrics as well as docker run-time metrics.}
%      \end{cvitems}
%    }
%---------------------------------------------------------
  \cventry
    %{Rule360 决策规则分析系统 - 高级软件工程师} % Job title
    {Decision Rule Analytics - Senior Data Engineer} % Job title
    {PayPal{\ \cdotp\ \ }Decision Analytics and Management Department} % Organization
    {Aug. 2017 - Now} % Location
    {\ } % Date(s)
    {
      \begin{cvitems} % Description(s) of tasks/responsibilities
        %\item {与全球分析师紧密合作,进行风控数据分析(SQL, Pandas),构建决策规则分析系统(Flask, React),包括规则解析(XML)、近实时决策监控、决策规则性能指标评估、决策因子引用分析(Neo4j)等模块,节省约80\%的决策分析成本}
        %\item {优化风控大数据结构化存储读写性能(MySQL),支持全量大数据更新}
        \item {Cooperated with global analysts and analyzed global transactions data (SQL, Pandas).}
        \item {Built Decision Rule Analytics System (Flask, React), including rule parsing (XML), near real time decision data monitoring, rule performance metrics evaluate, decision factors reference analysis (Neo4j). Saved about 80\% decision analytics cost.}
        \item {Optimized read/write performance on risk big data structural storage (MySQL), supported full data update.}
      \end{cvitems}
    }
    
%---------------------------------------------------------
  \cventry
    %{HyperNet 自动化风控平台 - 数据分析与全栈开发} % Job title
    {Automatic Risk Control Platform} % Job title Data Analysis and Full Stack Develop
    {} % Organization
    {} % Location
    {\ } % Date(s)
    {
      \begin{cvitems} % Description(s) of tasks/responsibilities
        %\item {构建统一数据预处理平台,对不同源数据(HDFS, Teradata)进行Spark ETL,进行特征统计分析与特征独热编码,整合生成训练与测试数据集}
        %\item {使用XGBoost构建决策树模型,按风控经验指标过滤决策规则生成风控方案;使用React构建前端损失报警与自动风控方案监控管理平台}
        %\item {构建SQL风控规则翻译引擎,支持用户从任意SQL规则翻译成IBM决策引擎规则}
        %\item {有效缩短2-3周生成风控方案的周期,节省约70\%人力成本,每月止损收益约\$80k,荣获PayPal Spot Award(5\%)}
        \item {Developed unified data preprocessing platform to support Spark ETL from diverse data source (HDFS, Teradata), and feature statistics and feature encoding.}
        \item {Built decision tree model by XGBoost, and set up performance metrics for generating accurate risk solutions.}
        \item {Built standard SQL to risk solution (IBM decision rules) translate engine.}
        \item {Developed loss alerting and risk solutions monitoring web platform by React/AntD.}
        \item {The project effectively cut down 2-3 weeks for generating risk solutions, saved about 70\% human resources and saved loss about \$80k per month. Won Paypal Spot Award (5\%).}
      \end{cvitems}
    }

%---------------------------------------------------------
  \cventry
    %{Spider 实时交易数据监控报警平台 - 后端开发} % Job title
    {Realtime Transaction Monitoring and Alerting} % Job title Backend Develop
    {} % Organization
    {} % Location
    {\ } % Date(s)
    {
      \begin{cvitems} % Description(s) of tasks/responsibilities
        %\item {开发报警任务执行系统, 消费交易决策引擎产生的警报(Apscheduler),并调度执行事后聚合分析(Pandas, Spark)、可视化(Matplotlib)、邮件报警(Jinja2)等任务}
        %\item {基于Celery, RabbitMQ实现任务链和异步优先级任务调度,提高报警服务性能及可靠性}
        \item {Developed alert action consumption module (Apscheduler) for Risk Decision Engine, and post alert concentration analysis (Pandas, Spark), visualization (Matplotlib) and mail alerting (Jinja2).}
        \item {Implemented task chains and asynchronized prioritized task scheduling (Celery, RabbitMQ), improved the alerting service performance and reliability.}
      \end{cvitems}
    }
    
%---------------------------------------------------------
  \cventry
    %{Mola大型分布式NoSQL数据库 - 研发工程师} % Job title
    {Mola Large Distributed NoSQL Database - Research and Develop Engineer} % Job title
    {Baidu{\ \cdotp\ \ }Department of Infrastructure (INF)} % Organization
    {Mar. 2017 - Jun. 2017} % Location
    {\ } % Date(s)
    {
      \begin{cvitems} % Description(s) of tasks/responsibilities
        %\item {研究Mola分布式架构,Meta Server选主机制(Lease),Node数据分片自动恢复}
        %\item {通过加锁解决数据存储节点Node中的分片在自动恢复过程中,异步修改Recover Task引发的恢复任务错误异常,降低Mola服务的故障率}
        \item {Research on Mola distributed database architecture, master election on metadata server (Lease), data fragment auto-recover on node server.}
        \item {Fixed and enhanced the asynchronized write in data fragment auto-recover process by applying lock mechanism.}
      \end{cvitems}
    }

%---------------------------------------------------------
  \cventry
    %{精益广告投放数据分析 - 数据工程师(实习)} % Job title
    {Advertise Big Data - Intern Data Engineer} % Job title
    {Baidu{\ \cdotp\ \ }Commercial Ecosystem Data Product Department} % Organization
    {Aug. 2016 - Feb. 2017} % Location
    {\ } % Date(s)
    {
      \begin{cvitems} % Description(s) of tasks/responsibilities
        %\item {精益对凤巢、Holmes等广告数据进行监测与评估,建立漏斗模型(人群转化、流动、归因分析)分析广告营销活动效果}
        %\item {负责对上游广告源数据进行ETL(Hive)及数据拼接、编写Spark UDF,优化ETL速率,并线上验证数据分析准确性}
        \item {Monitoring advertise big data from Fengchao and Holmes data source. Assisted in building funnel model for analyzing crowd transferring, flowing, and attribution analysis, to evaluate the effects of advertising and marketing campaigns.}
        \item {Assisted in upstream data ETL (Hive), implementing Spark UDF and optimizing ETL performance.}
      \end{cvitems}
    }
        
%---------------------------------------------------------
\end{cventries}



%%-------------------------------------------------------------------------------
%	SECTION TITLE
%-------------------------------------------------------------------------------
\cvsection{Work Experience}


%-------------------------------------------------------------------------------
%	CONTENT
%-------------------------------------------------------------------------------
\begin{cventries}

%---------------------------------------------------------
  \cventry
    {Software Architect} % Job title
    {Omnious. Co., Ltd.} % Organization
    {Seoul, S.Korea} % Location
    {Jun. 2017 - May. 2018} % Date(s)
    {
      \begin{cvitems} % Description(s) of tasks/responsibilities
        \item {Provisioned an easily managable hybrid infrastructure(Amazon AWS + On-premise) utilizing IaC(Infrastructure as Code) tools like Ansible, Packer and Terraform.}
        \item {Built fully automated CI/CD pipelines on CircleCI for containerized applications using Docker, AWS ECR and Rancher.}
        \item {Designed an overall service architecture and pipelines of the Machine Learning based Fashion Tagging API SaaS product with the micro-services architecture.}
        \item {Implemented several API microservices in Node.js Koa and in the serverless AWS Lambda functions.}
        \item {Deployed a centralized logging environment(ELK, Filebeat, CloudWatch, S3) which gather log data from docker containers and AWS resources.}
        \item {Deployed a centralized monitoring environment(Grafana, InfluxDB, CollectD) which gather system metrics as well as docker run-time metrics.}
      \end{cvitems}
    }

%---------------------------------------------------------
  \cventry
    {Co-founder \& Software Engineer} % Job title
    {PLAT Corp.} % Organization
    {Seoul, S.Korea} % Location
    {Jan. 2016 - Jun. 2017} % Date(s)
    {
      \begin{cvitems} % Description(s) of tasks/responsibilities
        \item {Implemented RESTful API server for car rental booking application(CARPLAT in Google Play).}
        \item {Built and deployed overall service infrastructure utilizing Docker container, CircleCI, and several AWS stack(Including EC2, ECS, Route 53, S3, CloudFront, RDS, ElastiCache, IAM), focusing on high-availability, fault tolerance, and auto-scaling.}
        \item {Developed an easy-to-use Payment module which connects to major PG(Payment Gateway) companies in Korea.}
      \end{cvitems}
    }

%---------------------------------------------------------
  \cventry
    {Software Engineer \& Security Researcher (Compulsory Military Service)} % Job title
    {R.O.K Cyber Command, MND} % Organization
    {Seoul, S.Korea} % Location
    {Aug. 2014 - Apr. 2016} % Date(s)
    {
      \begin{cvitems} % Description(s) of tasks/responsibilities
        \item {Lead engineer on agent-less backtracking system that can discover client device's fingerprint(including public and private IP) independently of the Proxy, VPN and NAT.}
        \item {Implemented a distributed web stress test tool with high anonymity.}
        \item {Implemented a military cooperation system which is web based real time messenger in Scala on Lift.}
      \end{cvitems}
    }

%---------------------------------------------------------
  \cventry
    {Game Developer Intern at Global Internship Program} % Job title
    {NEXON} % Organization
    {Seoul, S.Korea \& LA, U.S.A} % Location
    {Jan. 2013 - Feb. 2013} % Date(s)
    {
      \begin{cvitems} % Description(s) of tasks/responsibilities
        \item {Developed in Cocos2d-x an action puzzle game(Dragon Buster) targeting U.S. market.}
        \item {Implemented API server which is communicating with game client and In-App Store, along with two other team members who wrote the game logic and designed game graphics.}
        \item {Won the 2nd prize in final evaluation.}
      \end{cvitems}
    }

%---------------------------------------------------------
  \cventry
    {Software Engineer} % Job title
    {ShitOne Corp.} % Organization
    {Seoul, S.Korea} % Location
    {Dec. 2011 - Feb. 2012} % Date(s)
    {
      \begin{cvitems} % Description(s) of tasks/responsibilities
        \item {Developed a proxy drive smartphone application which connects proxy driver and customer.}
        \item {Implemented overall Android application logic and wrote API server for community service, along with lead engineer who designed bidding protocol on raw socket and implemented API server for bidding.}
      \end{cvitems}
    }

%---------------------------------------------------------
  \cventry
    {Freelance Penetration Tester} % Job title
    {SAMSUNG Electronics} % Organization
    {S.Korea} % Location
    {Sep. 2013, Mar. 2011 - Oct. 2011} % Date(s)
    {
      \begin{cvitems} % Description(s) of tasks/responsibilities
        \item {Conducted penetration testing on SAMSUNG KNOX, which is solution for enterprise mobile security.}
        \item {Conducted penetration testing on SAMSUNG Smart TV.}
      \end{cvitems}
      %\begin{cvsubentries}
      %  \cvsubentry{}{KNOX(Solution for Enterprise Mobile Security) Penetration Testing}{Sep. 2013}{}
      %  \cvsubentry{}{Smart TV Penetration Testing}{Mar. 2011 - Oct. 2011}{}
      %\end{cvsubentries}
    }

%---------------------------------------------------------
\end{cventries}

%-------------------------------------------------------------------------------
%	SECTION TITLE
%-------------------------------------------------------------------------------
\cvsection{Extracurriculum Experience}
%-------------------------------------------------------------------------------
%	CONTENT
%-------------------------------------------------------------------------------
\begin{cventries}
%---------------------------------------------------------
  \cventry
    %{Alexa智能家居项目 - 负责人} % Job title
    {Alexa Smart Home Group - Group Leader} % Job title
    {} % Organization
    {} % Location
    {Aug. 2018 - Oct. 2018} % Date(s)
    {
      \begin{cvitems} % Description(s) of tasks/responsibilities
        %\item {参与PayPal创新实验室智能家居组,负责Alexa与飞利浦智能灯具集成开发,组织创新日活动并发表Alexa与智能家居相关演讲}
        \item {Attended Paypal Innovation Lab Smart Home Group, lead to develop and integrate Alexa with Philips smart lamps. Organized innovation day activity and made a speech on smart home technology.}
      \end{cvitems}
    }

%---------------------------------------------------------
  \cventry
    %{量化投资策略分析 - 算法研究} % Job title
    {Quantitative Investment Analysis  - Algorithm Research} % Job title
    {} % Organization
    {} % Location
    {Feb. 2016 - Oct. 2016} % Date(s)
    {
      \begin{cvitems} % Description(s) of tasks/responsibilities
        %\item {基于python实现Hurst指数、SMA双均线突破、分类预测择时策略}
        %\item {研究并提出基于最大信息熵的股票特征选取方法,对比主成分分析、奇异值分解等方法高出约4.5\%的预测精确度;研究并提出SRA-Voting混合二元分类模型,对比支持向量机、随机森林、AdaBoost高出约3.125\%的预测精确度}
        %\item {Implemented Hurst Exponent, dual SMA and classification based timing strategy with Python.}
        \item {Researched and proposed Maximal Information Coefficient based stock feature selection method, achieving 4.5\% higher predicting precision compared with Principle Component Analysis and Singular Value Decomposition method.}
        \item {Designed an ensemble classification model named SRA-Voting, gaining 3.125\% higher predicting precision compared with Support Vector Machine and Fandom Forest.}
      \end{cvitems}
    }
    
%---------------------------------------------------------
%  \cventry
%    {实习软件工程师{\ \cdotp\ \ }东方名城社区系统} % Job title
%    {人行道网络信息技术有限公司} % Organization
%    {\ } % Location
%    {2015年6月 - 2015年12月} % Date(s)
%    {
%      \begin{cvitems} % Description(s) of tasks/responsibilities
%        \item {参与开发以五角场镇为试点的社区治理系统(社区治理服务、邻里社交、社区公共服务),主要负责系统需求分析、移动端原型设计、Android应用开发、部分Java Web后台服务(SSM)开发}
%      \end{cvitems}
%    }
    
%---------------------------------------------------------
\end{cventries}



%%-------------------------------------------------------------------------------
%	SECTION TITLE
%-------------------------------------------------------------------------------
\cvsection{Work Experience}


%-------------------------------------------------------------------------------
%	CONTENT
%-------------------------------------------------------------------------------
\begin{cventries}

%---------------------------------------------------------
  \cventry
    {Software Architect} % Job title
    {Omnious. Co., Ltd.} % Organization
    {Seoul, S.Korea} % Location
    {Jun. 2017 - May. 2018} % Date(s)
    {
      \begin{cvitems} % Description(s) of tasks/responsibilities
        \item {Provisioned an easily managable hybrid infrastructure(Amazon AWS + On-premise) utilizing IaC(Infrastructure as Code) tools like Ansible, Packer and Terraform.}
        \item {Built fully automated CI/CD pipelines on CircleCI for containerized applications using Docker, AWS ECR and Rancher.}
        \item {Designed an overall service architecture and pipelines of the Machine Learning based Fashion Tagging API SaaS product with the micro-services architecture.}
        \item {Implemented several API microservices in Node.js Koa and in the serverless AWS Lambda functions.}
        \item {Deployed a centralized logging environment(ELK, Filebeat, CloudWatch, S3) which gather log data from docker containers and AWS resources.}
        \item {Deployed a centralized monitoring environment(Grafana, InfluxDB, CollectD) which gather system metrics as well as docker run-time metrics.}
      \end{cvitems}
    }

%---------------------------------------------------------
  \cventry
    {Co-founder \& Software Engineer} % Job title
    {PLAT Corp.} % Organization
    {Seoul, S.Korea} % Location
    {Jan. 2016 - Jun. 2017} % Date(s)
    {
      \begin{cvitems} % Description(s) of tasks/responsibilities
        \item {Implemented RESTful API server for car rental booking application(CARPLAT in Google Play).}
        \item {Built and deployed overall service infrastructure utilizing Docker container, CircleCI, and several AWS stack(Including EC2, ECS, Route 53, S3, CloudFront, RDS, ElastiCache, IAM), focusing on high-availability, fault tolerance, and auto-scaling.}
        \item {Developed an easy-to-use Payment module which connects to major PG(Payment Gateway) companies in Korea.}
      \end{cvitems}
    }

%---------------------------------------------------------
  \cventry
    {Software Engineer \& Security Researcher (Compulsory Military Service)} % Job title
    {R.O.K Cyber Command, MND} % Organization
    {Seoul, S.Korea} % Location
    {Aug. 2014 - Apr. 2016} % Date(s)
    {
      \begin{cvitems} % Description(s) of tasks/responsibilities
        \item {Lead engineer on agent-less backtracking system that can discover client device's fingerprint(including public and private IP) independently of the Proxy, VPN and NAT.}
        \item {Implemented a distributed web stress test tool with high anonymity.}
        \item {Implemented a military cooperation system which is web based real time messenger in Scala on Lift.}
      \end{cvitems}
    }

%---------------------------------------------------------
  \cventry
    {Game Developer Intern at Global Internship Program} % Job title
    {NEXON} % Organization
    {Seoul, S.Korea \& LA, U.S.A} % Location
    {Jan. 2013 - Feb. 2013} % Date(s)
    {
      \begin{cvitems} % Description(s) of tasks/responsibilities
        \item {Developed in Cocos2d-x an action puzzle game(Dragon Buster) targeting U.S. market.}
        \item {Implemented API server which is communicating with game client and In-App Store, along with two other team members who wrote the game logic and designed game graphics.}
        \item {Won the 2nd prize in final evaluation.}
      \end{cvitems}
    }

%---------------------------------------------------------
  \cventry
    {Software Engineer} % Job title
    {ShitOne Corp.} % Organization
    {Seoul, S.Korea} % Location
    {Dec. 2011 - Feb. 2012} % Date(s)
    {
      \begin{cvitems} % Description(s) of tasks/responsibilities
        \item {Developed a proxy drive smartphone application which connects proxy driver and customer.}
        \item {Implemented overall Android application logic and wrote API server for community service, along with lead engineer who designed bidding protocol on raw socket and implemented API server for bidding.}
      \end{cvitems}
    }

%---------------------------------------------------------
  \cventry
    {Freelance Penetration Tester} % Job title
    {SAMSUNG Electronics} % Organization
    {S.Korea} % Location
    {Sep. 2013, Mar. 2011 - Oct. 2011} % Date(s)
    {
      \begin{cvitems} % Description(s) of tasks/responsibilities
        \item {Conducted penetration testing on SAMSUNG KNOX, which is solution for enterprise mobile security.}
        \item {Conducted penetration testing on SAMSUNG Smart TV.}
      \end{cvitems}
      %\begin{cvsubentries}
      %  \cvsubentry{}{KNOX(Solution for Enterprise Mobile Security) Penetration Testing}{Sep. 2013}{}
      %  \cvsubentry{}{Smart TV Penetration Testing}{Mar. 2011 - Oct. 2011}{}
      %\end{cvsubentries}
    }

%---------------------------------------------------------
\end{cventries}

%-------------------------------------------------------------------------------
%	SECTION TITLE
%-------------------------------------------------------------------------------
\cvsection{Education}
%-------------------------------------------------------------------------------
%	CONTENT
%-------------------------------------------------------------------------------
\begin{cventries}
%---------------------------------------------------------
  \cventry
    {M.Eng. in Software Engineering} % Job title
    {SJTU (Shanghai Jiao Tong University)} % Organization
    {Sep. 2014 - Mar. 2017} % Location
    {\ } % Date(s)
    {\ }

%---------------------------------------------------------
  \cventry
    {B.Eng. in Software Engineering} % Job title
    {UESTC (University of Electronic Science and Technology of China)} % Organization
    {Sep. 2010 - Jun. 2014} % Location
    {\ } % Date(s)
    {\ }
    
%---------------------------------------------------------
\end{cventries}



%%-------------------------------------------------------------------------------
%	SECTION TITLE
%-------------------------------------------------------------------------------
\cvsection{Skills}


%-------------------------------------------------------------------------------
%	CONTENT
%-------------------------------------------------------------------------------
\begin{cvskills}

%---------------------------------------------------------
  \cvskill
    {DevOps} % Category
    {AWS, Docker, Kubernetes, Rancher, Vagrant, Packer, Terraform, Jenkins, CircleCI} % Skills

%---------------------------------------------------------
  \cvskill
    {Back-end} % Category
    {Koa, Express, Django, REST API} % Skills

%---------------------------------------------------------
  \cvskill
    {Front-end} % Category
    {Hugo, Redux, React, HTML5, LESS, SASS} % Skills

%---------------------------------------------------------
  \cvskill
    {Programming} % Category
    {Node.js, Python, JAVA, OCaml, LaTeX} % Skills

%---------------------------------------------------------
  \cvskill
    {Languages} % Category
    {Korean, English, Japanese} % Skills

%---------------------------------------------------------
\end{cvskills}

%-------------------------------------------------------------------------------
%	SECTION TITLE
%-------------------------------------------------------------------------------
\cvsection{Skills}
%-------------------------------------------------------------------------------
%	CONTENT
%-------------------------------------------------------------------------------
\begin{cvskills}
%---------------------------------------------------------
  \cvskill
    {Programming} % Category
    {Python, C++, Java, SQL, Shell} % Skills
%---------------------------------------------------------
  \cvskill
    {Framework} % Category
    {Pandas, Flask, Celery, React, Spark} % Skills
%---------------------------------------------------------
  \cvskill
    {Tool} % Category
    {Teradata, MySQL, MongoDB, Emacs, Git, Jenkins, Docker, Excel, Tableau} % Skills
%---------------------------------------------------------
  \cvskill
    {English} % Category
    {CET6, GRE(314), can be used as work language} % Skills
%---------------------------------------------------------
\end{cvskills}



%%-------------------------------------------------------------------------------
%	SECTION TITLE
%-------------------------------------------------------------------------------
\cvsection{Extracurricular Activity}


%-------------------------------------------------------------------------------
%	CONTENT
%-------------------------------------------------------------------------------
\begin{cventries}

%---------------------------------------------------------
  \cventry
    {Core Member} % Affiliation/role
    {B10S (B1t 0n the Security, Underground hacker team)} % Organization/group
    {S.Korea} % Location
    {Nov. 2011 - PRESENT} % Date(s)
    {
      \begin{cvitems} % Description(s) of experience/contributions/knowledge
        \item {Gained expertise in penetration testing areas, especially targeted on web application and software.}
        \item {Participated on a lot of hacking competition and won a good award.}
        \item {Held several hacking competitions non-profit, just for fun.}
      \end{cvitems}
    }

%---------------------------------------------------------
  \cventry
    {Member} % Affiliation/role
    {WiseGuys (Hacking \& Security research group)} % Organization/group
    {S.Korea} % Location
    {Jun. 2012 - PRESENT} % Date(s)
    {
      \begin{cvitems} % Description(s) of experience/contributions/knowledge
        \item {Gained expertise in hardware hacking areas from penetration testing on several devices including wireless router, smartphone, CCTV and set-top box.}
        \item {Trained wannabe hacker about hacking technique from basic to advanced and ethics for white hackers by hosting annual Hacking Camp.}
      \end{cvitems}
    }

%---------------------------------------------------------
  \cventry
    {Core Member \& President at 2013} % Affiliation/role
    {PoApper (Developers' Network of POSTECH)} % Organization/group
    {Pohang, S.Korea} % Location
    {Jun. 2010 - Jun. 2017} % Date(s)
    {
      \begin{cvitems} % Description(s) of experience/contributions/knowledge
        \item {Reformed the society focusing on software engineering and building network on and off campus.}
        \item {Proposed various marketing and network activities to raise awareness.}
      \end{cvitems}
    }

%---------------------------------------------------------
  \cventry
    {Member} % Affiliation/role
    {PLUS (Laboratory for UNIX Security in POSTECH)} % Organization/group
    {Pohang, S.Korea} % Location
    {Sep. 2010 - Oct. 2011} % Date(s)
    {
      \begin{cvitems} % Description(s) of experience/contributions/knowledge
        \item {Gained expertise in hacking \& security areas, especially about internal of operating system based on UNIX and several exploit techniques.}
        \item {Participated on several hacking competition and won a good award.}
        \item {Conducted periodic security checks on overall IT system as a member of POSTECH CERT.}
        \item {Conducted penetration testing commissioned by national agency and corporation.}
      \end{cvitems}
    }

%---------------------------------------------------------
  \cventry
    {Member} % Affiliation/role
    {MSSA (Management Strategy Club of POSTECH)} % Organization/group
    {Pohang, S.Korea} % Location
    {Sep. 2013 - Jun. 2017} % Date(s)
    {
      \begin{cvitems} % Description(s) of experience/contributions/knowledge
        \item {Gained knowledge about several business field like Management, Strategy, Financial and marketing from group study.}
        \item {Gained expertise in business strategy areas and inisght for various industry from weekly industry analysis session.}
      \end{cvitems}
    }

%---------------------------------------------------------
\end{cventries}

%%-------------------------------------------------------------------------------
%	SECTION TITLE
%-------------------------------------------------------------------------------
\cvsection{Honors \& Awards}


%-------------------------------------------------------------------------------
%	SUBSECTION TITLE
%-------------------------------------------------------------------------------
\cvsubsection{International}


%-------------------------------------------------------------------------------
%	CONTENT
%-------------------------------------------------------------------------------
\begin{cvhonors}

%---------------------------------------------------------
  \cvhonor
    {Finalist} % Award
    {DEFCON 26th CTF Hacking Competition World Final} % Event
    {Las Vegas, U.S.A} % Location
    {2018} % Date(s)

%---------------------------------------------------------
  \cvhonor
    {Finalist} % Award
    {DEFCON 25th CTF Hacking Competition World Final} % Event
    {Las Vegas, U.S.A} % Location
    {2017} % Date(s)

%---------------------------------------------------------
  \cvhonor
    {Finalist} % Award
    {DEFCON 22nd CTF Hacking Competition World Final} % Event
    {Las Vegas, U.S.A} % Location
    {2014} % Date(s)

%---------------------------------------------------------
  \cvhonor
    {Finalist} % Award
    {DEFCON 21st CTF Hacking Competition World Final} % Event
    {Las Vegas, U.S.A} % Location
    {2013} % Date(s)

%---------------------------------------------------------
  \cvhonor
    {Finalist} % Award
    {DEFCON 19th CTF Hacking Competition World Final} % Event
    {Las Vegas, U.S.A} % Location
    {2011} % Date(s)

%---------------------------------------------------------
  \cvhonor
    {6th Place} % Award
    {SECUINSIDE Hacking Competition World Final} % Event
    {Seoul, S.Korea} % Location
    {2012} % Date(s)

%---------------------------------------------------------
\end{cvhonors}


%-------------------------------------------------------------------------------
%	SUBSECTION TITLE
%-------------------------------------------------------------------------------
\cvsubsection{Domestic}


%-------------------------------------------------------------------------------
%	CONTENT
%-------------------------------------------------------------------------------
\begin{cvhonors}

%---------------------------------------------------------
  \cvhonor
    {3rd Place} % Award
    {WITHCON Hacking Competition Final} % Event
    {Seoul, S.Korea} % Location
    {2015} % Date(s)

%---------------------------------------------------------
  \cvhonor
    {Silver Prize} % Award
    {KISA HDCON Hacking Competition Final} % Event
    {Seoul, S.Korea} % Location
    {2017} % Date(s)

%---------------------------------------------------------
  \cvhonor
    {Silver Prize} % Award
    {KISA HDCON Hacking Competition Final} % Event
    {Seoul, S.Korea} % Location
    {2013} % Date(s)

%---------------------------------------------------------
  \cvhonor
    {2nd Award} % Award
    {HUST Hacking Festival} % Event
    {S.Korea} % Location
    {2013} % Date(s)

%---------------------------------------------------------
  \cvhonor
    {3rd Award} % Award
    {HUST Hacking Festival} % Event
    {S.Korea} % Location
    {2010} % Date(s)

%---------------------------------------------------------
  \cvhonor
    {3rd Award} % Award
    {Holyshield 3rd Hacking Festival} % Event
    {S.Korea} % Location
    {2012} % Date(s)

%---------------------------------------------------------
  \cvhonor
    {2nd Award} % Award
    {Holyshield 3rd Hacking Festival} % Event
    {S.Korea} % Location
    {2011} % Date(s)

%---------------------------------------------------------
  \cvhonor
    {5th Place} % Award
    {PADOCON Hacking Competition Final} % Event
    {Seoul, S.Korea} % Location
    {2011} % Date(s)

%---------------------------------------------------------
\end{cvhonors}

%%-------------------------------------------------------------------------------
%	SECTION TITLE
%-------------------------------------------------------------------------------
\cvsection{Presentation}


%-------------------------------------------------------------------------------
%	CONTENT
%-------------------------------------------------------------------------------
\begin{cventries}

%---------------------------------------------------------
  \cventry
    {Presenter for <Hosting Web Application for Free utilizing GitHub, Netlify and CloudFlare>} % Role
    {DevFest Seoul by Google Developer Group Korea} % Event
    {Seoul, S.Korea} % Location
    {Nov. 2017} % Date(s)
    {
      \begin{cvitems} % Description(s)
        \item {Introduced the history of web technology and the JAM stack which is for the modern web application development.}
        \item {Introduced how to freely host the web application with high performance utilizing global CDN services.}
      \end{cvitems}
    }

%---------------------------------------------------------
  \cventry
    {Presenter for <DEFCON 20th : The way to go to Las Vegas>} % Role
    {6th CodeEngn (Reverse Engineering Conference)} % Event
    {Seoul, S.Korea} % Location
    {Jul. 2012} % Date(s)
    {
      \begin{cvitems} % Description(s)
        \item {Introduced CTF(Capture the Flag) hacking competition and advanced techniques and strategy for CTF}
      \end{cvitems}
    }

%---------------------------------------------------------
  \cventry
    {Presenter for <Metasploit 101>} % Role
    {6th Hacking Camp - S.Korea} % Event
    {S.Korea} % Location
    {Sep. 2012} % Date(s)
    {
      \begin{cvitems} % Description(s)
        \item {Introduced basic procedure for penetration testing and how to use Metasploit}
      \end{cvitems}
    }

%---------------------------------------------------------
\end{cventries}

%%-------------------------------------------------------------------------------
%	SECTION TITLE
%-------------------------------------------------------------------------------
\cvsection{Writing}


%-------------------------------------------------------------------------------
%	CONTENT
%-------------------------------------------------------------------------------
\begin{cventries}

%---------------------------------------------------------
  \cventry
    {Founder \& Writer} % Role
    {A Guide for Developers in Start-up} % Title
    {Facebook Page} % Location
    {Jan. 2015 - PRESENT} % Date(s)
    {
      \begin{cvitems} % Description(s)
        \item {Drafted daily news for developers in Korea about IT technologies, issues about start-up.}
      \end{cvitems}
    }

%---------------------------------------------------------
\end{cventries}

%%-------------------------------------------------------------------------------
%	SECTION TITLE
%-------------------------------------------------------------------------------
\cvsection{Program Committees}


%-------------------------------------------------------------------------------
%	CONTENT
%-------------------------------------------------------------------------------
\begin{cvhonors}

%---------------------------------------------------------
  \cvhonor
    {Problem Writer} % Position
    {2016 CODEGATE Hacking Competition World Final} % Committee
    {S.Korea} % Location
    {2016} % Date(s)

%---------------------------------------------------------
  \cvhonor
    {Organizer \& Co-director} % Position
    {1st POSTECH Hackathon} % Committee
    {S.Korea} % Location
    {2013} % Date(s)

%---------------------------------------------------------
\end{cvhonors}



%-------------------------------------------------------------------------------
\end{document}
