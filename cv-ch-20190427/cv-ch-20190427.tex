%!TEX TS-program = xelatex
%!TEX encoding = UTF-8 Unicode
% Awesome CV LaTeX Template for CV/Resume
%
% This template has been downloaded from:
% https://github.com/posquit0/Awesome-CV
%
% Author:
% Claud D. Park <posquit0.bj@gmail.com>
% http://www.posquit0.com
%
% Template license:
% CC BY-SA 4.0 (https://creativecommons.org/licenses/by-sa/4.0/)
%


%-------------------------------------------------------------------------------
% CONFIGURATIONS
%-------------------------------------------------------------------------------
% A4 paper size by default, use 'letterpaper' for US letter
\documentclass[11pt, a4paper, UTF8]{awesome-cv}

% Chinese
\usepackage[UTF8, heading = false, scheme = plain]{ctex}

% Configure page margins with geometry
\geometry{left=1.4cm, top=.8cm, right=1.4cm, bottom=1.8cm, footskip=.5cm}

% Specify the location of the included fonts
\fontdir[fonts/]

% Color for highlights
% Awesome Colors: awesome-emerald, awesome-skyblue, awesome-red, awesome-pink, awesome-orange
%                 awesome-nephritis, awesome-concrete, awesome-darknight
\colorlet{awesome}{awesome-skyblue}
% Uncomment if you would like to specify your own color
% \definecolor{awesome}{HTML}{CA63A8}

% Colors for text
% Uncomment if you would like to specify your own color
% \definecolor{darktext}{HTML}{414141}
% \definecolor{text}{HTML}{333333}
% \definecolor{graytext}{HTML}{5D5D5D}
% \definecolor{lighttext}{HTML}{999999}

% Set false if you don't want to highlight section with awesome color
\setbool{acvSectionColorHighlight}{false}

% If you would like to change the social information separator from a pipe (|) to something else
\renewcommand{\acvHeaderSocialSep}{\quad\textbar\quad}

\renewcommand{\acvHeaderAfterNameSkip}{1mm}
\renewcommand{\acvHeaderAfterSocialSkip}{1mm}

%\renewcommand*{\headerfirstnamestyle}[1]{{\fontsize{32pt}{1em}\headerfontlight\color{graytext} #1}}
\renewcommand*{\headerlastnamestyle}[1]{{\fontsize{26pt}{1em}\headerfont\bfseries\color{text} #1}}
\renewcommand*{\headerpositionstyle}[1]{{\fontsize{9pt}{1em}\bodyfont\itshape\color{awesome} #1}}
%\renewcommand*{\headeraddressstyle}[1]{{\fontsize{8pt}{1em}\headerfont\itshape\color{text} #1}}
\renewcommand*{\headersocialstyle}[1]{{\fontsize{8pt}{1em}\headerfont\color{text} #1}}
\renewcommand*{\headerquotestyle}[1]{{\fontsize{8pt}{1em}\bodyfont\itshape\color{text} #1}}

\renewcommand*{\entrylocationstyle}[1]{{\fontsize{9pt}{1em}\bodyfontlight\slshape\color{text} #1}}

%-------------------------------------------------------------------------------
%	PERSONAL INFORMATION
%	Comment any of the lines below if they are not required
%-------------------------------------------------------------------------------
% Available options: circle|rectangle,edge/noedge,left/right
\photo[circle,noedge,left]{./profile}
\name{}{杨健}
%\position{Software Architect{\enskip\cdotp\enskip}Security Expert}
\position{软件工程师{\enskip\cdotp\enskip}上海}
%\address{42-8, Bangbae-ro 15-gil, Seocho-gu, Seoul, 00681, Rep. of KOREA}

\mobile{(+86) 18516191607}
\email{ernestyj@outlook.com}
%\homepage{www.posquit0.com}
\github{Ernestyj}
%\linkedin{posquit0}
% \gitlab{gitlab-id}
% \stackoverflow{SO-id}{SO-name}
% \twitter{@twit}
% \skype{skype-id}
% \reddit{reddit-id}
% \medium{madium-id}
% \googlescholar{googlescholar-id}{name-to-display}
%% \firstname and \lastname will be used
% \googlescholar{googlescholar-id}{}
% \extrainfo{extra informations}

\quote{硕士{\ \ \cdotp\ \ }上海交通大学{\quad\textbar\quad}学士{\ \ \cdotp\ \ }电子科技大学}


%-------------------------------------------------------------------------------
\begin{document}

% Print the header with above personal informations
% Give optional argument to change alignment(C: center, L: left, R: right)
\makecvheader[R]

% Print the footer with 3 arguments(<left>, <center>, <right>)
% Leave any of these blank if they are not needed
\makecvfooter
  {\ }
  {\ }
  {\thepage}


%-------------------------------------------------------------------------------
%	CV/RESUME CONTENT
%	Each section is imported separately, open each file in turn to modify content
%-------------------------------------------------------------------------------

%-------------------------------------------------------------------------------
%	SECTION TITLE
%-------------------------------------------------------------------------------
\cvsection{个人简介}

%-------------------------------------------------------------------------------
%	CONTENT
%-------------------------------------------------------------------------------
\begin{cvparagraph}
%---------------------------------------------------------
Current Site Reliability Engineer at start-up company Kasa. 7+ years experience specializing in the backend development, infrastructure automation, and computer hacking/security. Super nerd who loves Vim, Linux and OS X and enjoys to customize all of the development environment. Interested in devising a better problem-solving method for challenging tasks, and learning new technologies and tools if the need arises.
\end{cvparagraph}



%\input{cv/experience.tex}
%-------------------------------------------------------------------------------
%	SECTION TITLE
%-------------------------------------------------------------------------------
\cvsection{工作经历}
%-------------------------------------------------------------------------------
%	CONTENT
%-------------------------------------------------------------------------------
\begin{cventries}
%---------------------------------------------------------
%  \cventry
%    {Software Architect} % Job title
%    {Omnious. Co., Ltd.} % Organization
%    {Seoul, S.Korea} % Location
%    {Jun. 2017 - May. 2018} % Date(s)
%    {
%      \begin{cvitems} % Description(s) of tasks/responsibilities
%        \item {Provisioned an easily managable hybrid infrastructure(Amazon AWS + On-premise) utilizing IaC(Infrastructure as Code) tools like Ansible, Packer and Terraform.}
%        \item {Built fully automated CI/CD pipelines on CircleCI for containerized applications using Docker, AWS ECR and Rancher.}
%        \item {Designed an overall service architecture and pipelines of the Machine Learning based Fashion Tagging API SaaS product with the micro-services architecture.}
%        \item {Implemented several API microservices in Node.js Koa and in the serverless AWS Lambda functions.}
%        \item {Deployed a centralized logging environment(ELK, Filebeat, CloudWatch, S3) which gather log data from docker containers and AWS resources.}
%        \item {Deployed a centralized monitoring environment(Grafana, InfluxDB, CollectD) which gather system metrics as well as docker run-time metrics.}
%      \end{cvitems}
%    }
%---------------------------------------------------------
  \cventry
    {数据工程师} % Job title
    {PayPal{\ \cdotp\ \ }全球风险控制} % Organization
    {上海} % Location
    {2017年3月 - 2017年7月} % Date(s)
    {
      \begin{cvitems} % Description(s) of tasks/responsibilities
        \item {Mola是百度内部广泛使用的NoSQL数据库,也为百度云提供全托管NoSQL数据库服务}
        \item {Built and deployed overall service infrastructure utilizing Docker container, CircleCI, and several AWS stack(Including EC2, ECS, Route 53, S3, CloudFront, RDS, ElastiCache, IAM), focusing on high-availability, fault tolerance, and auto-scaling.}
        \item {Developed an easy-to-use Payment module which connects to major PG(Payment Gateway) companies in Korea.}
      \end{cvitems}
    }
    
%---------------------------------------------------------
  \cventry
    {项目} % Job title
    {} % Organization
    {} % Location
    {\ } % Date(s)
    {
      \begin{cvitems} % Description(s) of tasks/responsibilities
        \item {Implemented RESTful API server for car rental booking application(CARPLAT in Google Play).}
      \end{cvitems}
    }
    
%---------------------------------------------------------
  \cventry
    {研发工程师{\ \cdotp\ \ }Mola分布式数据库} % Job title
    {百度{\ \cdotp\ \ }基础架构部} % Organization
    {上海} % Location
    {2017年3月 - 2017年7月} % Date(s)
    {
      \begin{cvitems} % Description(s) of tasks/responsibilities
        \item {Mola是百度内部广泛使用的NoSQL数据库,也为百度云提供全托管NoSQL数据库服务}
        \item {在Proxyserver中为百度云公网访问MolaDB服务添加标准访问日志,添加请求key记录,有效缩短后台故障排查时间}
        \item {解决Node中分片恢复过程Task修改异步引发恢复错误问题,有效降低了Mola服务的故障率}
      \end{cvitems}
    }

%---------------------------------------------------------
  \cventry
    {实习数据工程师{\ \cdotp\ \ }精益-广告投放数据分析} % Job title
    {百度{\ \cdotp\ \ }生态数据产品部} % Organization
    {上海} % Location
    {2016年7月 - 2016年10月} % Date(s)
    {
      \begin{cvitems} % Description(s) of tasks/responsibilities
        \item {精益对凤巢、Holmes等广告数据进行监测与评估,建立漏斗模型(人群转化、流动、归因分析)分析广告营销活动效果}
        \item {对上游广告源数据进行ETL(Hive/SQL)及数据拼接、编写UDF(Spark),优化ETL速率,并线上验证数据分析准确性}
        \item {接入按查询/搜索关键词新建模型(Spark/Scala)的功能业务,广告主可以按关键词查看广告投放效果}
      \end{cvitems}
    }
        
%---------------------------------------------------------
  \cventry
    {实习算法工程师{\ \cdotp\ \ }量化投资策略分析} % Job title
    {荣石投资管理有限公司} % Organization
    {上海} % Location
    {2017年3月 - 2017年7月} % Date(s)
    {
      \begin{cvitems} % Description(s) of tasks/responsibilities
        \item {基于python实现Hurst指数、SMA双均线突破、分类预测择时策略}
        \item {从Wind获取原始股票交易数据,并构建交易时间序列数据库(Cassandra),作为公司内部数据源便于后续数据分析}
        \item {提出基于最大信息熵的股票特征选取方法,对比主成分分析、奇异值分解等方法高出约4.5\%的预测精确度}
        \item {提出SRA-Voting混合二元分类模型,对比SVM、随机森林、AdaBoost高出约3.125\%的预测精确度}
      \end{cvitems}
    }
    
%---------------------------------------------------------
\end{cventries}



%\input{cv/skills.tex}
%-------------------------------------------------------------------------------
%	SECTION TITLE
%-------------------------------------------------------------------------------
\cvsection{技能}
%-------------------------------------------------------------------------------
%	CONTENT
%-------------------------------------------------------------------------------
\begin{cvskills}
%---------------------------------------------------------
  \cvskill
    {编程语言} % Category
    {C++, Python, Java, React} % Skills
%---------------------------------------------------------
  \cvskill
    {工具} % Category
    {Teradata, Excel, Tableau} % Skills
%---------------------------------------------------------
  \cvskill
    {英文} % Category
    {CET6/GRE/良好听说读写能力} % Skills
%---------------------------------------------------------
\end{cvskills}



%\input{cv/extracurricular.tex}
%\input{cv/honors.tex}
%\input{cv/presentation.tex}
%\input{cv/writing.tex}
%\input{cv/committees.tex}


%-------------------------------------------------------------------------------
\end{document}
