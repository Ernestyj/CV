%!TEX TS-program = xelatex
%!TEX encoding = UTF-8 Unicode
% Awesome CV LaTeX Template for CV/Resume
%
% This template has been downloaded from:
% https://github.com/posquit0/Awesome-CV
%
% Author:
% Claud D. Park <posquit0.bj@gmail.com>
% http://www.posquit0.com
%
% Template license:
% CC BY-SA 4.0 (https://creativecommons.org/licenses/by-sa/4.0/)
%


%-------------------------------------------------------------------------------
% CONFIGURATIONS
%-------------------------------------------------------------------------------
% A4 paper size by default, use 'letterpaper' for US letter
\documentclass[11pt, a4paper, UTF8]{awesome-cv}

% Chinese
\usepackage[UTF8, heading = false, scheme = plain]{ctex}

% Configure page margins with geometry
\geometry{left=1.4cm, top=.8cm, right=1.4cm, bottom=1.8cm, footskip=.5cm}

% Specify the location of the included fonts
\fontdir[fonts/]

% Color for highlights
% Awesome Colors: awesome-emerald, awesome-skyblue, awesome-red, awesome-pink, awesome-orange
%                 awesome-nephritis, awesome-concrete, awesome-darknight
\colorlet{awesome}{awesome-skyblue}
% Uncomment if you would like to specify your own color
% \definecolor{awesome}{HTML}{CA63A8}

% Colors for text
% Uncomment if you would like to specify your own color
% \definecolor{darktext}{HTML}{414141}
% \definecolor{text}{HTML}{333333}
% \definecolor{graytext}{HTML}{5D5D5D}
% \definecolor{lighttext}{HTML}{999999}

% Set false if you don't want to highlight section with awesome color
\setbool{acvSectionColorHighlight}{false}

% If you would like to change the social information separator from a pipe (|) to something else
\renewcommand{\acvHeaderSocialSep}{\quad\textbar\quad}

%%%%%%%%%%%%%%%%%% custom settings

\geometry{left=2.0cm, top=1.5cm, right=2.0cm, bottom=2.0cm, footskip=.5cm}

\renewcommand{\acvHeaderAfterNameSkip}{1mm}
\renewcommand{\acvHeaderAfterSocialSkip}{1mm}

\renewcommand*{\paragraphstyle}{\fontsize{10pt}{1em}\bodyfontlight\upshape\color{text}}

%\renewcommand*{\headerfirstnamestyle}[1]{{\fontsize{32pt}{1em}\headerfontlight\color{graytext} #1}}
\renewcommand*{\headerlastnamestyle}[1]{{\fontsize{26pt}{1em}\headerfont\bfseries\color{text} #1}}
\renewcommand*{\headerpositionstyle}[1]{{\fontsize{10pt}{1em}\bodyfont\itshape\color{awesome} #1}}
%\renewcommand*{\headeraddressstyle}[1]{{\fontsize{8pt}{1em}\headerfont\itshape\color{text} #1}}
\renewcommand*{\headersocialstyle}[1]{{\fontsize{10pt}{1em}\headerfont\color{text} #1}}
\renewcommand*{\headerquotestyle}[1]{{\fontsize{10pt}{1em}\bodyfont\itshape\color{text} #1}}

\renewcommand*{\entrytitlestyle}[1]{{\fontsize{11pt}{1em}\bodyfont\bfseries\color{darktext} #1}}
\renewcommand*{\entrypositionstyle}[1]{{\fontsize{10pt}{1em}\bodyfont\itshape\color{graytext} #1}}
\renewcommand*{\entrydatestyle}[1]{{\fontsize{9pt}{1em}\bodyfont\itshape\color{graytext} #1}}
\renewcommand*{\entrylocationstyle}[1]{{\fontsize{9pt}{1em}\bodyfontlight\itshape\color{text} #1}}
\renewcommand*{\descriptionstyle}[1]{{\fontsize{10pt}{1em}\bodyfontlight\upshape\color{text} #1}}

\renewcommand*{\skilltypestyle}[1]{{\fontsize{10pt}{1em}\bodyfont\bfseries\color{darktext} #1}}
\renewcommand*{\skillsetstyle}[1]{{\fontsize{10pt}{1em}\bodyfontlight\color{text} #1}}

%%%%%%%%%%%%%%%%%% custom settings

%-------------------------------------------------------------------------------
%	PERSONAL INFORMATION
%	Comment any of the lines below if they are not required
%-------------------------------------------------------------------------------
% Available options: circle|rectangle,edge/noedge,left/right
%\photo[circle,noedge,left]{./profile}
\name{}{杨健}
%\position{Software Architect{\enskip\cdotp\enskip}Security Expert}
\position{软件工程师{\enskip\cdotp\enskip}上海}
%\address{42-8, Bangbae-ro 15-gil, Seocho-gu, Seoul, 00681, Rep. of KOREA}

\mobile{(+86) 18516191607}
\email{ernestyj@outlook.com}
%\homepage{www.posquit0.com}
\github{Ernestyj}
%\linkedin{posquit0}
% \gitlab{gitlab-id}
% \stackoverflow{SO-id}{SO-name}
% \twitter{@twit}
% \skype{skype-id}
% \reddit{reddit-id}
% \medium{madium-id}
% \googlescholar{googlescholar-id}{name-to-display}
%% \firstname and \lastname will be used
% \googlescholar{googlescholar-id}{}
% \extrainfo{extra informations}

\quote{硕士{\ \ \cdotp\ \ }上海交通大学{\quad\textbar\quad}学士{\ \ \cdotp\ \ }电子科技大学{\quad\textbar\quad}软件工程专业}


%-------------------------------------------------------------------------------
\begin{document}

% Print the header with above personal informations
% Give optional argument to change alignment(C: center, L: left, R: right)
\makecvheader[R]

% Print the footer with 3 arguments(<left>, <center>, <right>)
% Leave any of these blank if they are not needed
\makecvfooter
  {\ }
  {\ }
  {\thepage}


%-------------------------------------------------------------------------------
%	CV/RESUME CONTENT
%	Each section is imported separately, open each file in turn to modify content
%-------------------------------------------------------------------------------

%-------------------------------------------------------------------------------
%	SECTION TITLE
%-------------------------------------------------------------------------------
\cvsection{个人简介}

%-------------------------------------------------------------------------------
%	CONTENT
%-------------------------------------------------------------------------------
\begin{cvparagraph}
%---------------------------------------------------------
两年半工作经验,有基于C++的大型分布式NoSQL数据库项目开发经验。熟悉使用Python开发后台服务、自动化系统,进行数据分析、建模与可视化,有基于React的前端开发经验。作为主力工程师参与设计、开发金融风控决策规则分析系统、自动化风控解决方案生成平台、实时交易数据监控报警平台。
\end{cvparagraph}



%\input{cv/experience.tex}
%-------------------------------------------------------------------------------
%	SECTION TITLE
%-------------------------------------------------------------------------------
\cvsection{工作经历}
%-------------------------------------------------------------------------------
%	CONTENT
%-------------------------------------------------------------------------------
\begin{cventries}
%---------------------------------------------------------
%  \cventry
%    {Software Architect} % Job title
%    {Omnious. Co., Ltd.} % Organization
%    {Seoul, S.Korea} % Location
%    {Jun. 2017 - May. 2018} % Date(s)
%    {
%      \begin{cvitems} % Description(s) of tasks/responsibilities
%        \item {Provisioned an easily managable hybrid infrastructure(Amazon AWS + On-premise) utilizing IaC(Infrastructure as Code) tools like Ansible, Packer and Terraform.}
%        \item {Built fully automated CI/CD pipelines on CircleCI for containerized applications using Docker, AWS ECR and Rancher.}
%        \item {Designed an overall service architecture and pipelines of the Machine Learning based Fashion Tagging API SaaS product with the micro-services architecture.}
%        \item {Implemented several API microservices in Node.js Koa and in the serverless AWS Lambda functions.}
%        \item {Deployed a centralized logging environment(ELK, Filebeat, CloudWatch, S3) which gather log data from docker containers and AWS resources.}
%        \item {Deployed a centralized monitoring environment(Grafana, InfluxDB, CollectD) which gather system metrics as well as docker run-time metrics.}
%      \end{cvitems}
%    }
%---------------------------------------------------------
  \cventry
    {Rule360 决策规则分析系统 - 后端开发} % Job title
    {PayPal{\ \cdotp\ \ }数据分析与决策管理部门} % Organization
    {2017年8月 - 至今} % Location
    {\ } % Date(s)
    {
      \begin{cvitems} % Description(s) of tasks/responsibilities
        \item {与分析师紧密合作,进行风控数据分析(SQL, Pandas),构建决策规则分析系统(Flask, React),包括规则解析、近实时决策监控、决策规则性能指标评估、决策因子引用分析等模块,节省约80\%的决策分析成本}
        %\item {参与设计、构建从损失报警到生成解决方案的自动化风控平台HyperNet,自动生成的风控方案有效性不低于人工解决方案,缩短2-3周生成风控方案的周期,节省约70\%人力成本,每月止损收益约\$80k}
        %\item {独立构建了一站式客户服务指标分析系统Lexa360,帮助提高自助服务推荐解决方案的准确率,提升客户服务满意度,减少客户联系率,降低人工客服成本}
      \end{cvitems}
    }
    
%---------------------------------------------------------
  \cventry
    {HyperNet 自动化风控平台 - 前后端设计与开发} % Job title
    {} % Organization
    {} % Location
    {\ } % Date(s)
    {
      \begin{cvitems} % Description(s) of tasks/responsibilities
        \item {构建统一数据预处理平台,对不同源数据(HDFS, Teradata)进行Spark ETL,进行特征统计分析与特征编码,整合生成训练与测试数据集}
        \item {使用XGBoost构建决策树模型,按风控经验指标过滤决策规则生成风控方案;使用React构建前端损失报警与自动风控方案监控管理平台}
        \item {有效缩短2-3周生成风控方案的周期,节省约70\%人力成本,每月止损收益约\$80k,荣获PayPal Spot Award}
      \end{cvitems}
    }

%---------------------------------------------------------
  \cventry
    {Spider 实时交易数据监控报警平台 - 后端开发} % Job title
    {} % Organization
    {} % Location
    {\ } % Date(s)
    {
      \begin{cvitems} % Description(s) of tasks/responsibilities
        \item {开发报警任务执行系统, 消费交易决策引擎产生的警报(Apscheduler),并调度执行事后聚合分析(Pandas, Spark)、决策反馈、可视化(Matplotlib)、邮件报警等任务}
        \item {基于Celery, RabbitMQ实现任务链和异步优先级任务调度,提供更高效稳定的报警服务}
      \end{cvitems}
    }
    
%---------------------------------------------------------
  \cventry
    {Mola大型分布式NoSQL数据库 - 研发工程师} % Job title
    {百度{\ \cdotp\ \ }基础架构部} % Organization
    {2017年3月 - 2017年7月} % Location
    {\ } % Date(s)
    {
      \begin{cvitems} % Description(s) of tasks/responsibilities
        \item {在Proxyserver中为百度云公网访问MolaDB服务添加标准访问日志(C++),添加请求key记录,有效缩短后台故障排查时间}
        \item {解决Node中分片恢复过程Task修改异步引发恢复错误问题,有效降低了Mola服务的故障率}
      \end{cvitems}
    }

%---------------------------------------------------------
  \cventry
    {精益广告投放数据分析 - 实习数据工程师} % Job title
    {百度{\ \cdotp\ \ }生态数据产品部} % Organization
    {2016年7月 - 2016年10月} % Location
    {\ } % Date(s)
    {
      \begin{cvitems} % Description(s) of tasks/responsibilities
        \item {精益对凤巢、Holmes等广告数据进行监测与评估,建立漏斗模型(人群转化、流动、归因分析)分析广告营销活动效果}
        \item {对上游广告源数据进行ETL(Hive)及数据拼接、编写Spark UDF,优化ETL速率,并线上验证数据分析准确性}
        %\item {接入按查询/搜索关键词新建模型(Spark)的功能业务,广告主可以按关键词查看广告投放效果}
      \end{cvitems}
    }
        
%---------------------------------------------------------
\end{cventries}



%\input{cv/experience.tex}
%-------------------------------------------------------------------------------
%	SECTION TITLE
%-------------------------------------------------------------------------------
\cvsection{项目经历}
%-------------------------------------------------------------------------------
%	CONTENT
%-------------------------------------------------------------------------------
\begin{cventries}
%---------------------------------------------------------
  \cventry
    {Alexa智能家居项目} % Job title
    {} % Organization
    {} % Location
    {2018年8月 - 2018年10月} % Date(s)
    {
      \begin{cvitems} % Description(s) of tasks/responsibilities
        \item {PayPal创新实验室智能家居组负责人,参与Alexa与飞利浦智能灯具集成开发,组织创新日活动并发表Alexa与智能家居相关演讲}
      \end{cvitems}
    }

%---------------------------------------------------------
  \cventry
    {量化投资策略分析} % Job title
    {} % Organization
    {} % Location
    {2016年2月 - 2016年10月} % Date(s)
    {
      \begin{cvitems} % Description(s) of tasks/responsibilities
        \item {基于python实现Hurst指数、SMA双均线突破、分类预测择时策略}
        \item {提出基于最大信息熵的股票特征选取方法,对比主成分分析、奇异值分解等方法高出约4.5\%的预测精确度}
        \item {提出SRA-Voting混合二元分类模型,对比SVM、随机森林、AdaBoost高出约3.125\%的预测精确度}
      \end{cvitems}
    }
    
%---------------------------------------------------------
%  \cventry
%    {实习软件工程师{\ \cdotp\ \ }东方名城社区系统} % Job title
%    {人行道网络信息技术有限公司} % Organization
%    {\ } % Location
%    {2015年6月 - 2015年12月} % Date(s)
%    {
%      \begin{cvitems} % Description(s) of tasks/responsibilities
%        \item {参与开发以五角场镇为试点的社区治理系统(社区治理服务、邻里社交、社区公共服务),主要负责系统需求分析、移动端原型设计、Android应用开发、部分Java Web后台服务(SSM)开发}
%      \end{cvitems}
%    }
    
%---------------------------------------------------------
\end{cventries}



%\input{cv/skills.tex}
%-------------------------------------------------------------------------------
%	SECTION TITLE
%-------------------------------------------------------------------------------
\cvsection{技能}
%-------------------------------------------------------------------------------
%	CONTENT
%-------------------------------------------------------------------------------
\begin{cvskills}
%---------------------------------------------------------
  \cvskill
    {编程语言} % Category
    {Python, C++, Java, SQL, Shell} % Skills
%---------------------------------------------------------
  \cvskill
    {框架技术} % Category
    {Pandas, Flask, Celery, React, Spark} % Skills
%---------------------------------------------------------
  \cvskill
    {工具} % Category
    {Teradata, MySQL, MongoDB, Emacs, Git, Jenkins, Docker, Excel, Tableau} % Skills
%---------------------------------------------------------
  \cvskill
    {英文} % Category
    {CET6, GRE, 良好听说读写能力} % Skills
%---------------------------------------------------------
\end{cvskills}



%\input{cv/extracurricular.tex}
%\input{cv/honors.tex}
%\input{cv/presentation.tex}
%\input{cv/writing.tex}
%\input{cv/committees.tex}


%-------------------------------------------------------------------------------
\end{document}
